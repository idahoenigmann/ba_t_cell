\chapter{T-Cells, Calcium Concentration}
\label{chapter:t-cell}

Lymphocytes form a key component of the immune system. T cells are one type of lymphocyte and are responsible for responding to pathogens, allergens and tumors. Different subtypes of t cells exist, that fulfill various responsibilities. They are transported throughout the body via the lymphatic system and blood.\cite{Kumar2018}

Precursor cells are formed in the bone marrow. Once they are transported to the thymus they undergo maturation and selection to become t cells. Each cell forms receptors, called t cell receptors (TCR), that respond to one perticular out of many ($10^6 - 10^9$) possible major histocompatibility complex (MHC) present on antigens and antigen presenting cells (APC). Important aspects of the selection are ensuring that the t cells react to foreign MHCs, but not to those present on the body's own cells.\cite{Ashby2024}

In positive selection cells in the thymus present short pieces of proteins, called peptides, on their MHC. If a t cell is unable to bind, it will undergo apoptosis, a type of cell death. T cells which were able to bind recieve survival signals. Negative selection verifies that t cells will not attack the body's own cells. This is done by only selecting t cells which only bind moderatly to the peptides presented, as a strong bond sugessts that these t cells would have a high likelihood of being reactive to own cells.\cite{Hagel2018} If a t cell passed both the positive and negative selection it is transported to the periphery.

There are multiple types of peripheral t cells. Native t cells respond to new antigens. Cytotoxic t cells kill cells with a MHC compatible with their TCR. Helper T cells activate other parts of the immune response. Memory t cells shorten the reaction time when the same antigen is encountered again at a later point in time. Suppressor t cells moderate the immune response.\cite{Ganong1997}

\section{Components of a T-Cell}

calcium storage, membrane, APC docking point

\section{Activation}

what does activation mean? how are the calcium levels affected?

