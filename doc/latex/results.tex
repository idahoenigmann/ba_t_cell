\chapter{Results}
\label{chapter:results}

This chapter gives answers to the questions from chapter~\ref{chapter:introduction}. To answer the methods described in this work, such as approximation, clustering and outlier detection, are combined and used.

\section{Proposed algorithm for Detecting Activated T Cells}
\label{sec:proposed-algorithm}

The first and most relevant question was ''Which criteria can distinguish between unactivated, activated and pre-activated cells?''.

First we give a method for filtering the pre-activated cells from a dataset. From their nature we expect a high value in \Calcium concentration at the start of the recording. Using the approximation from chapter~\ref{chapter:approximating} it is easy to get the approximate \Calcium concentration value at the start of the recording, as it is the parameter $u$. Using the algorithm~\ref{alg:outlier_detection} with parameters threshold as $[\infty, 0.5]$ and parameters\_used as [$u$] gives good results. It returns the indices of particles, which are pre-activated in the data sets of the positive controls.

After having filtered out pre-activated particles, we want to distinguish between unactivated and activated particles. For this we propose the following steps:

\begin{enumerate}
	\item get positive control, negative control and experiment recordings
	\item transform each particle time series of all three data sets to the parameter list by approximating it with a combination of sigmoid functions, according to chapter~\ref{chapter:approximating}, using the algorithm \ref{alg:main}
	\item use outlier detection, which is described in algorithm~\ref{alg:outlier_detection}, to filter out non-conforming cells from both the positive and negative control, as well as pre-activated cells, and particles where the approximation yielded suboptimal results
	\item use clustering method, as one described in chapter~\ref{chapter:clustering}, with parameters of filtered negative and positive control as input to get the clustering parameters
	\item predict the membership of the experiment particle parameters to the clusters to get a prediction of activation
\end{enumerate}

This algorithm is implemented in Python and shown in appendix~\ref{chapter:python_implementation}.

[TODO results of applying it to experiment data]

The algorithm can be adapted by using different clustering methods, or specifying other methods of separating the particles based on the parameters derived.


\section{Types of Activated Cells}

Question: Do different types of activated cells exists? How are they different?

Answer: Apply Gaussian Mixture Clustering to activated cells only. (Separate human and mouse cells, otherwise two clusters of eben das)

\section{Oscillation in Decrease}

Question: With which frequencies does the Calcium concentration repeat after activation?

Answer: Results from frequency analysis

\section{Difference between Mouse and Human Cells}

Question: Is there a difference in frequencies between mouse and human cells?

Answer: Compare mean and covariance between the two.