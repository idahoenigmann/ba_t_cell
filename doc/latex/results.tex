\chapter{Results}
\label{chapter:results}

\section{Proposed algorithm for Detecting Activated T Cells}

Question: Which criteria can distinguish between unactivated, activated and pre-activated cells?

Answer for activated and unactivated cells:

\begin{enumerate}
	\item get positive control, negative control and experiment recordings
	\item get parameters of each particle in all data sets using approximation described above
	\item use outlier detection to filter out non-conforming cells form both the positive and negative control
	\item use Gaussian Mixture Model with input parameters of filtered negative and positive control to get means and covariances of two clusters
	\item predict the zugehörigkeit of the experiment particle parameters to the clusters to get a prediction of activation
\end{enumerate}

For pre-activated cell detection:

Filter out those with too high $u$ or to small $w_1$ using the outlier detection.

\section{Types of Activated Cells}

Question: Do different types of activated cells exists? How are they different?

Answer: Apply Gaussian Mixture Clustering to activated cells only. (Separate human and mouse cells, otherwise two clusters of eben das)

\section{Oscillation on Decrease}

Question: With which frequencies does the Calcium concentration repeat after activation?

Answer: Results from frequency analysis

\section{Difference between Mouse and Human Cells}

Question: Is there a difference in frequencies between mouse and human cells?

Answer: Compare mean and covariance between the two.