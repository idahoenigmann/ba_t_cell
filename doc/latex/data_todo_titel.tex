\chapter{Data}
\label{chapter:data}

calcium concentration shows activatedness of t cells (reference chapter t cells), relativly easy to measure

\section{Structure of Data}

what format is the data in? which columns are present + datatypes

\begin{table}[h!]
	\centering
	\begin{tabular}{|c|c|l|}
		\hline
		\textbf{Name} & \textbf{Data Type} & \textbf{Description} \\
		\hline
		x & float64 & Position of cell in pixels along the horizontal axis \\
		\hline
		y & float64 & Position of cell in pixels along the vertical axis \\
		\hline
		frame & int32 & Number of frame, with frame rate of 1 frame per second \\
		\hline
		mass short & float64 & Brightness of cell in 340nm channel \\
		\hline
		bg short & float64 & Background in 340nm channel \\
		\hline
		mass long & float64 & Brightness of cell in 380nm channel \\
		\hline
		bg long & float64 & Background in 380nm channel \\
		\hline
		ratio & float64 & Calculated as mass short divided by mass long \\
		\hline
		particle & int32 & Identification for each particle \\
		\hline
	\end{tabular}
	\caption{Description and data type of all columns present in the data matrix.}
\end{table}


\section{How it was generated}

exprimental setup, what types of t cells where used?, apc layer, explain steps in experiment

\begin{itemize}
	\item Date: 18/12/23
	\item Cells:  Jurkat wt labelled with Fura-2
	\item Sample: PDMS coated with OKT3 (positive control)
	\item Imaging: SDT3, ratiometric Ca imaging, 340nm \& 380 nm, Total cycle time 1000ms (-> 1 frame per sec in sum/ratio image)
	\item pixel size: 1.6 um / px
\end{itemize}


\subsection{Measuring Calcium Concentration}

how is the calcium concentration measured? different wavelengths and then ratio between them, show example video frame

\subsection{Processing}

tracking of particles (in sum of two images), numbering them, removing bad ones (too out of focus, too short)
